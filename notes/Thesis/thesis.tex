\documentclass[a4paper, 12pt, twoside]{report}
\usepackage[english]{babel}
\usepackage[top=4cm,bottom=4cm,left=3cm,right=3cm,asymmetric]{geometry}
\usepackage[utf8]{inputenc}
\usepackage{amsmath, amssymb, amsfonts, amsthm, dsfont}
\usepackage{relsize}
\usepackage{tikz}
\usetikzlibrary{arrows.meta, calc, positioning}
\usepackage{float}
\usepackage{caption}
\usepackage{fancyhdr}
\usepackage{bibgerm} 
\usepackage{cite}
\usepackage[hidelinks]{hyperref}
\usepackage[all]{hypcap}
\usepackage{subcaption} 
\usepackage{setspace}
\usepackage{mdframed}
\captionsetup[figure]{labelfont=it, font=footnotesize}
\captionsetup[subfigure]{labelfont=it,font=footnotesize}

\theoremstyle{plain}
\newtheorem{theorem}{Theorem}[chapter]
\newtheorem{lemma}[theorem]{Lemma}
\newtheorem{corollary}[theorem]{Corollary}
\newtheorem{definition}[theorem]{Definition}

\newcommand*\ruleline[1]{\par\noindent\raisebox{.8ex}{\makebox[\linewidth]{\hrulefill\hspace{1ex}\raisebox{-.8ex}{#1}\hspace{1ex}\hrulefill}}}


\parindent0em

\begin{document}
	
	\begin{titlepage}
		\centering
		{\large \scshape Rheinisch-Westfälische Technische Hochschule Aachen}\\
		{\large
			Chair for Software Modeling and Verification\\
			Prof. Dr. Ir. Dr. h. c. Joost-Pieter Katoen\\}
		\vspace*{\fill}
		{\large \ruleline{Master Thesis}}\\
		\vspace{1cm}
		\textbf{\Huge Comparing Hierarchical and On-The-Fly Model Checking for Java Pointer Programs}\\
		\vspace{1cm}
		\hrule
		\vspace{1cm}
		{\Large \textbf{Sally Chau} \\}
		\vspace{0.25cm}
		{\large Matriculation Number 370584} \\
		%\vspace{0.3cm}
		{\large \today}\\
		\vspace{3cm}
		{\large First Reviewer: apl. Prof. Dr. Thomas Noll \\ Second Reviewer: Prof. Dr. Ir. Dr. h. c. Joost-Pieter Katoen \\ Supervisor: Christoph Matheja} \\ 
		\vspace{1cm}
		\vspace*{\fill}
	\end{titlepage}
	
	\pagestyle{empty}
	
	\clearpage\mbox{}\clearpage
	
	\chapter*{Acknowledgement} 
	
	
	\clearpage\mbox{}\clearpage
	
	\chapter*{Eidesstattliche Erklärung}
	
	Hiermit versichere ich an Eides statt und durch meine Unterschrift, dass die vorliegende Arbeit von mir selbstständig, ohne fremde Hilfe angefertigt worden ist. Inhalte und Passagen, die aus fremden Quellen stammen und direkt oder indirekt übernommen worden sind, wurden als solche kenntlich gemacht. Ferner versichere ich, dass ich keine andere, außer der im Literaturverzeichnis angegebenen Literatur verwendet habe. Die Arbeit wurde bisher keiner Prüfungsbehörde vorgelegt und auch noch nicht veröffentlicht.
	\vspace{20 mm}
	
	\noindent\line(1,0){250}\\
	Bonn, den 28. September 2015, Sally Chau
	
	\clearpage\mbox{}\clearpage
	
	\chapter*{Abstract}
	
	\clearpage\mbox{}\clearpage
	
	\doublespacing
	\tableofcontents
	\singlespacing
	\clearpage\mbox{}\clearpage
	\thispagestyle{empty} 
	
	\pagestyle{fancy}
	\fancyhead[RE]{\nouppercase\leftmark}
	\fancyhead[LO]{\nouppercase\rightmark}
	\fancyhead[LE,RO]{\thepage}
	\cfoot{}
	
	
	\chapter{Introduction}
	
	\section{Attestor}
	% include current model checking approach (only checking main state space)
	
	\section{Related Work}
	
	\chapter{Preliminaries}
	
	\section{Notation}
	
	\section{Hypergraphs}
	\subsection{Hyperedge Replacement Grammars}
	
	\section{Model Checking}
	% Introduction: what is model checking? want to verify path properties for state spaces generated by attestor
	% define transition system, paths, property, state, satisfies
	
	\subsection{Linear Temporal Logic}
	%
	
	% reference book "Principles of Model Checking"
	 
	\textit{Linear Temporal Logic} (LTL) can be used to describe properties of paths in a transition system $S$. LTL formulae mainly consist of three components: the boolean operators \textit{negation} ($\neg$) and \textit{conjunction} ($\wedge$), the temporal operators \textit{next} ($\bigcirc$) and \textit{until} ($U$), and a set of atomic proposition $AP$, which contains the state labels of a transition system. A state label is an assertion about the value of a state, e.g., "$i=1$". 
	
	\begin{definition}[Syntax of LTL \cite{baier2008principles}]\label{def:ltl_syntax}
		Given a set $AP$ of atomic propositions with $a \in AP$, \textup{LTL formulae} are recursively defined as
		\begin{equation*}		
			\psi ::= \texttt{\textup{true}} \mid a \mid \neg \varphi \mid \varphi_1 \wedge \varphi_2 \mid \bigcirc \varphi \mid \varphi_1 \textbf{\textup{U}} \varphi_2.
		\end{equation*}
	\end{definition}

	% describe every formula type and when they are fulfilled by a path -> model relation
	% include graphics for each formula
	
	% introduce other operators expressed by the basic ones
	% include graphics
	
	\begin{definition}[Positive Normal Form \cite{baier2008principles}]\label{def:ltl_pnf}
		Given a set $AP$ of atomic propositions with $a \in AP$, LTL formulae in \textup{positive normal form} (PNF) are defined as
		\begin{equation*}		
		\varphi ::= \texttt{\textup{true}} \mid \texttt{\textup{false}} \mid a \mid \neg a \mid \varphi_1 \wedge \varphi_2 \mid \varphi_1 \vee \varphi_2 \mid \bigcirc \varphi \mid \varphi_1 \textbf{\textup{U}} \varphi_2 \mid \varphi_1 \textbf{\textup{R}} \varphi_2.
		\end{equation*}
	\end{definition}

	%explanation of PNF
	% every LTL formula can be expressed as PNF
	
	%example for LTL
	
	\subsection{LTL Model Checking}
	%Tableau method
	
	\section{Recursive State Machines} 
	% reference paper "Analysis of Recursive States Machines"
		\begin{definition}[Recursive State Machine \cite{alur2001analysis}]\label{def:rsm}
			A \textup{recursive state machine} (RSM) $A$ over a finite alphabet $\Sigma$ is given by a tuple $(A_1, ..., A_k)$, where each \textup{component state machine} (CSM) $A_i = (N_i \cup B_i, Y_i, En_i, Ex_i, \delta_i)$, $1 \leq i \leq k$, consists of
			\begin{itemize}
				\item a set $N_i$ of \textup{nodes} and a (disjoint) set $B_i$ of \textup{boxes},
				\item a \textup{labeling} $Y_i: B_i \mapsto \{1, ..., k\}$ that assigns to every box an index $j \in \{1, ..., k\}$ referring to one of the component state machines $A_1, ..., A_k$,
				\item a set of \textup{entry nodes} $En_i \subseteq N_i$,
				\item a set of \textup{exit nodes} $Ex_i \subseteq N_i$, and
				\item a \textup{transition relatio}n $\delta_i$, where transitions are of the form $(u, \sigma, v)$, where 
				\begin{itemize}
					\item the source $u$ is either a node of $N_i$ or a pair $(b, x)$, where $b$ is a box in $B_i$ and $x$ is an exit node in $Ex_j$ for $j = Y_i(b)$,
					\item the label $\sigma$ is in $\Sigma$, and 
					\item the destination $v$ is either a node in $N_i$ or a pair $(b, e)$, where $b$ is a box in $B_i$ and $e$ is an entry node in $En_j$ for $j = Y_i(b)$.
				\end{itemize}
			\end{itemize}
		\end{definition}
	
	%some explanations
	
	% example RSM
	\begin{figure}[!h]
		\begin{center}
			%\resizebox{0.7\textwidth}{!}{
				\begin{tikzpicture}
			%	\draw [gray, line width=0.05pt] (0.1,0.1) rectangle (0.2,0.2);
				[node distance=2cm,thick,
				normal/.style={circle, draw, fill=white, thick, font=\sffamily\Large\bfseries},
				rectstyle/.style={draw, minimum width=7.5cm, minimum height=4.5cm, rounded corners=1.5pt, fill=gray!5}]
		
				\node (a1) at (-3.5, 2.5) {$A_1$};
				\node (rect1) at (0,0) [rectstyle] {};
				\node (u1) at (-3.75,1) [normal, label=180:$u_1$] {};
				\node (u2) at (-3.75,-1) [normal, label=180:$u_2$] {};
				\node (u3) at (-2,-1) [normal, label=-90:$u_3$] {};
				\node (u4) at (3.75, 0) [normal, label=0:$u_4$] {};
				%boxes
				\node (b1) at (0,1) [draw,thick,minimum width=2.5cm,minimum height=1.5cm,fill=gray!20,rounded corners=1.5pt] {$b_1:A_2$};
				\node (b11) at (-1.25,0.75) [normal, minimum size=0.2cm, inner sep=0pt] {};
				\node (b12) at (-1.25,1.25) [normal, minimum size=0.2cm, inner sep=0pt] {};
				\node (b13) at (1.25,0.75) [normal, minimum size=0.2cm, inner sep=0pt] {};
				\node (b14) at (1.25,1.25) [normal, minimum size=0.2cm, inner sep=0pt] {};
				\node (b2) at (1,-1) [draw,thick,minimum width=2.5cm,minimum height=1.5cm,fill=gray!20,rounded corners=1.5pt] {$b_2:A_3$};
				\node (b21) at (-0.25,-1) [normal, minimum size=0.2cm, inner sep=0pt] {};
				\node (b22) at (2.25,-1) [normal, minimum size=0.2cm, inner sep=0pt] {};
				% arrows
				\draw[->,shorten >=0.5pt,out=0,in=180] (u1) to (b12);
				\draw[->,shorten >=0.5pt,out=0,in=180] (u2) to (u3);
				\draw[->,shorten >=0.5pt,out=0,in=180] (u3) to (b21);
				\draw[->,shorten >=0.5pt,out=0,in=225] (b22) to (u4);
				\draw[->,shorten >=0.5pt,out=0,in=135] (b14) to (u4);
				\draw[->,shorten >=0.5pt,out=-45,in=225,distance=1.5cm] (b13) to (b11);
				
				\node (a2) at (-3.5, -3) {$A_2$};
				\node (rect2) at (0,-5.5) [rectstyle] {};
				\node (v1) at (-3.75,-4.5) [normal, label=180:$v_1$] {};
				\node (v2) at (-3.75,-6.5) [normal, label=180:$v_2$] {};
				\node (v3) at (3.75, -4.5) [normal, label=0:$v_3$] {};
				\node (v4) at (3.75,-6.5) [normal, label=0:$v_4$] {};
				%boxes
				\node (c1) at (0,-4.5) [draw,thick,minimum width=2.5cm,minimum height=1.5cm,fill=gray!20,rounded corners=1.5pt] {$c_1:A_2$};
				\node (c11) at (-1.25,-4.25) [normal, minimum size=0.2cm, inner sep=0pt] {};
				\node (c12) at (-1.25,-4.75) [normal, minimum size=0.2cm, inner sep=0pt] {};
				\node (c13) at (1.25,-4.25) [normal, minimum size=0.2cm, inner sep=0pt] {};
				\node (c14) at (1.25,-4.75) [normal, minimum size=0.2cm, inner sep=0pt] {};
				\node (c2) at (0,-6.5) [draw,thick,minimum width=2.5cm,minimum height=1.5cm,fill=gray!20,rounded corners=1.5pt] {$c_2:A_3$};
				\node (c21) at (-1.25,-6.5) [normal, minimum size=0.2cm, inner sep=0pt] {};
				\node (c22) at (1.25,-6.5) [normal, minimum size=0.2cm, inner sep=0pt] {};
				% arrows
				\draw[->,shorten >=0.5pt,out=0,in=180] (v1) to (c11);
				\draw[->,shorten >=0.5pt,out=45,in=180] (v2) to (c12);
				\draw[->,shorten >=0.5pt,out=-45,in=180] (v2) to (c21);
				\draw[->,shorten >=0.5pt,out=45,in=225,distance=2.75cm] (c22) to (c12);
				\draw[->,shorten >=0.5pt,out=0,in=-135] (c22) to (v4);
				\draw[->,shorten >=0.5pt,out=0,in=135] (c13) to (v4);
				\draw[->,shorten >=0.5pt,out=0,in=180] (c14) to (v3);
				
				\node (a3) at (-3.5, -8.5) {$A_3$};
				\node (rect3) at (0,-11) [rectstyle] {};
				\node (w1) at (-3.75,-11) [draw, thick, fill=white, circle, label=180:$w_1$] {};
				\node (w2) at (3.75,-11) [draw, thick, fill=white, circle, label=0:$w_2$] {};
				%boxes
				\node (d) at (0,-11) [draw, thick, fill=gray!20,minimum width=2.5cm,minimum height=1.5cm,rounded corners=1.5pt] {$d:A_1$};
				\node (d1) at (-1.25,-10.75) [draw, thick, fill=white, circle, minimum size=0.2cm, inner sep=0pt] {};
				\node (d2) at (-1.25,-11.25) [draw, thick, fill=white, circle, minimum size=0.2cm, inner sep=0pt] {};
				\node (d3) at (1.25,-11) [draw, thick, fill=white, circle, minimum size=0.2cm, inner sep=0pt] {};
				% arrows
				\draw[->,shorten >=0.5pt,out=0,in=180] (w1) to (d1);
				\draw[->,shorten >=0.5pt,out=0,in=135] (d3) to (w2);
				\draw[->,shorten >=0.5pt,out=-45,in=-135] (w1) to (w2);
				
				\end{tikzpicture}%}
			\caption{A sample recursive state machine. Adopted from \cite{alur2001analysis}. Describe more....}\label{fig:rsm}
		\end{center}
	\end{figure}
	
	\subsection{Semantics}
	% semantics of RSM
	This section describes the global relation between component state machines $A_i$ of an RSM $A=(A_1, ..., A_k)$ in order to define its execution. A \textit{global state} of an RSM consists of boxes and nodes of its CSMs.
	
	\begin{definition}[(Global) State \cite{alur2001analysis}]\label{def:rsm_semantics}
		A \textup{(global) state} of an RSM $A=(A_1, ..., A_k)$ is a tuple $(b_1, ..., b_r, u)$, where $b_1, ..., b_r$ are boxes and $u$ is a node. The set $Q$ of global states of $A$ is $B^*N$, where $B=\bigcup_iB_i$ and $N=\bigcup_iN_i$. A state $(b_1, ..., b_r, u)$ with $b_i \in {B_j}_i$ for $1 \leq i \leq r$ and $u \in N_j$ is \textup{well-formed} if ${Y_j}_i(b_i) = j_{i+1}$ for $1 \leq i < r$ and ${Y_j}_r(b_r) = j$.
	\end{definition}

	%A state $(b_1, ..., b_r, u)$ of an RSM $A=(A_1, ..., A_k)$ can also be viewed as a string. 
	% example here
	
	A well-formed state $(b_1, ..., b_r, u)$ of an RSM $A=(A_1, ..., A_k)$ corresponds to a path through the components $A_j$ of $A$, where we enter component $A_j$ via box $b_r$ of component ${A_j}_r$.
	
	% example
	
	In order to transition between global states of an RSM $A$, we require the notion of a \textit{global transition relation} $\delta$ which enables us to not only transition between states within a CSM $A_j$ as defined by its transition relation $\delta_j$, but also between pairs of CSMs.
	
	\begin{definition}[(Global) Transition Relation \cite{alur2001analysis}]\label{def:rsm_transitionRelation}
		Let $s=(b_1, ..., b_r, u) \in Q$ be a state with $u\in N_j$ and $b_r \in B_m$ for an RSM $A=(A_1, ..., A_k)$. A \textup{(global) transition relation} $\delta$ for $A$ defines $(s, \sigma, s') \in \delta$ if and only if one of the following holds:
		\begin{enumerate}
			\item $(u, \sigma, u') \in \delta_j$ for a node $u'$ of $A_j$ and $s'=(b_1, ..., b_r, u')$.
			\item $(u, \sigma, (b',e))\in \delta_j$ for a box $b'$ of $A_j$ and $s'=(b_1, ..., b_r, b', e)$.
			\item $u$ is an exit-node of $A_j$, $((b_r, u), \sigma, u') \in \delta_m$ for a node $u'$ of $A_m$, and $s'=(b_1, ..., b_{r-1}, u')$.
			\item $u$ is an exit-node of $A_j$, $((b_r, u), \sigma, (b',e)) \in \delta_m$ for a box $b'$ of $A_m$, and $s'=(b_1, ..., b_{r-1}, b', e)$.
		\end{enumerate}
	\end{definition}

	Definition~\ref{def:rsm_transitionRelation} defines the possible kinds of transitions between global states $s, s' \in Q$ of an RSM $A$. Case 1 describes the scenario where the source and the destination states are both within the same component $A_j$, while case 2 depicts that a new component is entered via a box $b'$ of $A_j$. Thus, the current node of the destination state $s'$ is the entry-node $e$. Case 3 and 4 are both exiting component $A_j$ via the exit-node $u$. While case 3 returns to component $A_m$, from where we entered $A_j$ before, case 4 directly enters a new component via box $b'$ of component $A_m$.
	
	% depict cases graphically here
	
	After defining the terms of global states and the global transition relation for an RSM $A$, we can summarize these components together with the finite alphabet $\Sigma$ within the concept of a \textit{labeled transition system} $T_A$, which encodes the execution of $A$.
	
	\begin{definition}[Labeled Transition System \cite{alur2001analysis}]\label{def:rsm_transitionSystem}
		For an RSM $A=(A_1, ..., A_k)$, the \textup{labeled transition system} (LTS) $T_A=(Q, \Sigma, \delta)$ consists of
		\begin{itemize}
			\item a set of global states $Q$, 
			\item a finite alphabet $\Sigma$, and
			\item a global transition relation $\delta$.
		\end{itemize}		
		The LTS of an RSM $A$ is also called the \textup{unfolding} of $A$.
	\end{definition}
	
	\chapter{Hierarchical Model Checking with Recursive State Machines}	
	% briefly describe Büchi automaton and why it is not practical
	\section{Algorithm}
	\section{Implementation}
	\section{Evaluation}
	
	
	\chapter{On-The-Fly Hierarchical Model Checking}
	\section{Algorithm}
	\section{Implementation}
	\section{Evaluation}
	
	\chapter{Benchmarks}	
	\section{Experimental Setup}
	Describe Technical details here
	\section{Instances}
	Describe code examples and properties here
	\section{Result}
	Table of values
	
	\chapter{Conclusion}
	\section{Discussion}	
	\section{Outlook}
	
		- hierarchical failure trace and counter example generation, spuriosity
	- hybrid method between on-the-fly and RSM
	
	\bibliographystyle{gerplain}
	\bibliography{lit}{}
	
\end{document}